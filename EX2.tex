\documentclass{article}
\title{EX2}
\author{Joey}
\date{\today}
\begin{document}
\maketitle

\section{Task 1}

\begin{equation}
    Q=c m \Delta T
\end{equation}

\begin{equation}
    c(water) = 4.2 \times 10^{3} \mathrm{~J} /\left(\mathrm{kg}^{\circ} \mathrm{C}\right)
\end{equation}

\begin{equation}
    P_{loss} = P_{water} = 2600MW
\end{equation}

\begin{equation}
    m_{per second} = 6.180 \times 10^{3} \mathrm{~kg} /\mathrm{s}
\end{equation}

\section{Task 3}

RT-02 Boiling Water Reactors P30

\section{Task 4}

$1ppb=1/1000ppm$

$10 ppm = 200 * 50 ppb$

\section{Task 5}

A power station with cooling tower loses 600 kg/s of cooling water by evaporation.
The river water contains 0.6 g/l of calcium bicarbonate (hardness).
It loses 1200L/s water, gets 0.72kg/s,  $2.27 \times 10^{7} \mathrm{kg} / year$ of calcium bicarbonate (hardness).

$1 year = 31536000 s$

$1 kg = 2 L$

calcium bicarbonate:
$\mathrm{Ca}\left(\mathrm{HCO}_{3}\right)_{2}$
Molar mass: 162.1146 g/mol

slaked lime, aka Calcium hydroxide:
$\mathrm{Ca}(\mathrm{OH})_{2}$
Molar mass: 74.093 g/mol
Density: 2.21 g/cm


In a year of continuous operation,


\end{document}